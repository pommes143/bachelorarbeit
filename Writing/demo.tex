\documentclass[a4paper,11pt,oneside]{memoir}

\ifpdf
\usepackage[utf8]{inputenc}
\usepackage[T1]{fontenc}
\fi

\usepackage[english]{babel}

% This is the only file needed for actually using the smart-thesis style.
%
% Smart Thesis LaTeX template
%
% [To underline the amateurish flavour of this template, let's start with some
% nifty ASCII-Art (http://www.network-science.de/ascii/)...]
%
%      #######
%    /       ###
%   /         ##                                          #
%   ##        #                                          ##
%    ###                                                 ##
%   ## ###      ### /### /###     /###   ###  /###     ########
%    ### ###     ##/ ###/ /##  / / ###  / ###/ #### / ########
%      ### ###    ##  ###/ ###/ /   ###/   ##   ###/     ##
%        ### /##  ##   ##   ## ##    ##    ##            ##
%          #/ /## ##   ##   ## ##    ##    ##            ##
%           #/ ## ##   ##   ## ##    ##    ##            ##
%            # /  ##   ##   ## ##    ##    ##            ##
%  /##        /   ##   ##   ## ##    /#    ##            ##
% /  ########/    ###  ###  ### ####/ ##   ###           ##
%/     #####       ###  ###  ### ###   ##   ###           ##
%|
% \)
%
%
%  /###           /  /
% /  ############/ #/                              #
%/     #########   ##                             ###
%#     /  #        ##                              #
% ##  /  ##        ##
%    /  ###        ##  /##      /##       /###   ###        /###
%   ##   ##        ## / ###    / ###     / #### / ###      / #### /
%   ##   ##        ##/   ###  /   ###   ##  ###/   ##     ##  ###/
%   ##   ##        ##     ## ##    ### ####        ##    ####
%   ##   ##        ##     ## ########    ###       ##      ###
%    ##  ##        ##     ## #######       ###     ##        ###
%     ## #      /  ##     ## ##              ###   ##          ###
%      ###     /   ##     ## ####    /  /###  ##   ##     /###  ##
%       ######/    ##     ##  ######/  / #### /    ### / / #### /
%         ###       ##    ##   #####      ###/      ##/     ###/
%                         /
%                        /
%                       /
%                      /
%
% About:
% ======
%
% This template is a re-implementation of the "classicthesis" template by André
% Miede. However, it uses the "memoir" class as a basis, eliminating most
% of the external packages required by "classicthesis" and thus (hopefully)
% achieving a higher compatibility with other LaTeX packages. Large parts of
% the code have been adapted (ransacked) from André Miedes code.
%
% You can find more information about "classicthesis" at CTAN:
%
% https://www.ctan.org/pkg/classicthesis
%
% More information about the "memoir" package can be found here:
%
% https://www.ctan.org/pkg/memoir
%
%
% Authors:
% ========
%
% Jan Philip Göpfert, Andreas Stöckel
%
%
% License:
% ========
%
% This LaTeX template is published under the Creative Commons Zero license. To
% the extent possible under law, the authors have waived all copyright and
% related neighboring rights to Smart Thesis. This work is published from:
% Germany.
%


%
% EXTERNAL PACKAGES
%
\usepackage[usenames,dvipsnames,svgnames]{xcolor}
\usepackage{ragged2e} % for marginnote

%
% FONT DEFINITIONS
%

% Use the following code for XeLaTeX
\ifxetex

% Setup microtype for XeLaTeX
\usepackage[expansion=false,final]{microtype}

% Load the texgyrepagella font as main font and set the ``mathpazo'' font as
% math font
\usepackage{fontspec}
\usepackage{mathpazo}
\setmainfont%
	[%
		BoldFont       = texgyrepagella-bold.otf,
		ItalicFont     = texgyrepagella-italic.otf,
		BoldItalicFont = texgyrepagella-bolditalic.otf,
		Ligatures      = TeX
	]%
	{texgyrepagella-regular.otf}

% Define a new 'widespaced' font that is used in section headings
% or chapter titles. For backwards compatibility reasons this uses
% the old calling convention \newfontfamily\fontname[<options>{font}
% instead of the newer \newfontfamily\fontname{font}[<options>]
\newfontfamily\widespaced[%
	LetterSpace=15,%
	WordSpace={1.25, 1.25, 1.25}%
]{texgyrepagella-regular.otf}%

\DeclareRobustCommand{\spacedallcaps}[1]{%
	{\widespaced\oldstylenums{\MakeTextUppercase{#1}}}%
}%
\DeclareRobustCommand{\spacedlowsmallcaps}[1]{%
	{\widespaced\oldstylenums{\textsc{\MakeTextLowercase{#1}}}}%
}%
\fi

% Use the following code for pdfLaTex
\ifpdf

% Setup microtype for pdfLaTex
\usepackage[final,tracking=true,kerning=true,spacing=nonfrench]{microtype}


\usepackage[sc]{mathpazo}

\DeclareRobustCommand{\spacedallcaps}[1]{\textls[160]{\oldstylenums{\textsc{\MakeTextUppercase{#1}}}}}%
\DeclareRobustCommand{\spacedlowsmallcaps}[1]{\textls[80]{\oldstylenums{\textsc{\MakeTextLowercase{#1}}}}}%
\fi

% Increase linespread for Palatino
\renewcommand{\linespread}{1.05}

%
% COLOR DEFINITIONS
%

\definecolor{smartgray}{gray}{0.55}
\definecolor{smartred}{HTML}{800000}
\definecolor{smartplum}{HTML}{47294F}
\definecolor{smartblue}{HTML}{193A6B}
\definecolor{smartorange}{HTML}{AA4C00}
\definecolor{smartgreen}{HTML}{3C7705}
\definecolor{smartbutter}{HTML}{A08300}

\definecolor{tangored}{HTML}{CC0000}
\definecolor{tangoplum}{HTML}{75507B}
\definecolor{tangoblue}{HTML}{3465A4}
\definecolor{tangoorange}{HTML}{F57900}
\definecolor{tangogreen}{HTML}{73D216}
\definecolor{tangobutter}{HTML}{EDD400}

\definecolor{shadecolor}{gray}{0.95}

\colorlet{heading}{smartgray}
\colorlet{ref}{smartblue}
\colorlet{cite}{smartgreen}
\colorlet{url}{smartred}
\colorlet{caption}{black}
\colorlet{margincaption}{caption}

%
% SECTIONING STYLES (toledo)
%

% See http://tex.stackexchange.com/questions/161577/memoir-hangnum-chapter-number-in-right-margin-switch for the inspiration

\setsecnumdepth{subsection}

\makeatletter
\makechapterstyle{toledo}{%
	\chapterstyle{default}

	\renewcommand*{\chapterheadstart}{\vspace*{0.0cm}}

	\renewcommand*{\chapnumfont}{\normalfont\color{heading}\addfontfeature{SizeFeatures={Size=110}}}
	\newfont{\chapterNumber}{eurb10 scaled 7000}
	\renewcommand*{\chapnumfont}{\chapterNumber\color{heading}}
	\renewcommand*{\printchaptername}{}
	\renewcommand*{\chapternamenum}{}
	\renewcommand*{\printchapternum}{%
		\raisebox{0pt}[0pt][0pt]{%
		\makebox[0pt][l]{%
		\makebox[\dimexpr\textwidth+5.45em\relax][l]{%
			\parbox[t]{\textwidth}{\mbox{}}%
			\parbox[t]{5.45em}{\hfill\chapnumfont \thechapter}}}}}%
	\renewcommand*{\chapternamenum}{}
	\renewcommand*{\afterchapternum}{}

	\renewcommand*{\afterchaptertitle}{%
	\par\parbox{\textwidth}{\hrulefill}\par%
	\vspace*{4ex}}
	\renewcommand*{\chaptitlefont}{\normalfont\large}
	\renewcommand*{\printchaptertitle}[1]{%
		\parbox[b]{\textwidth}{%
			\raggedright\chaptitlefont\spacedallcaps{##1}%
		}%
	}

	\setsecheadstyle{\spacedlowsmallcaps}
	\setsecindent{-3.5ex}
	\setbeforesecskip{-4ex plus -1ex minus -.2ex}
	\setaftersecskip{4ex plus 1ex minus .2ex}

	\setsubsecheadstyle{\itshape}
	\setbeforesubsecskip{-4ex plus -1ex minus -.2ex}
	\setaftersubsecskip{4ex plus 1ex minus .2ex}
	\setsecnumformat{\spacedlowsmallcaps{\csname the##1\endcsname\quad}}

%	\setparaheadstyle{\normalsize\itshape}

	\renewcommand*{\abstractnamefont}{\spacedlowsmallcaps}
}
\makeatother

\chapterstyle{toledo}

%
% FIGURES
%

\captionstyle{\small}
\captionnamefont{\slshape\color{caption}}

\newsubfloat{figure}% Create subfloat in figure environment
\newsubfloat{table}% Create subfloat in table environment

%
% ACKNOWLEDGEMENT
%

\makeatletter
\newcommand{\acknowledgementname}{Acknowledgements}
\newenvironment{acknowledgement}{%
  \setup@bstract
  \if@bsrunin\else
    \ifnumber@bs \num@bs \else
      \begin{\absnamepos}\abstractnamefont\acknowledgementname\end\absnamepos%
      \vspace{\abstitleskip}%
    \fi
  \fi
  \put@bsintoc%
  \begin{@bstr@ctlist}\if@bsrunin\@bsrunintitle\fi\abstracttextfont}%
  {\par\end{@bstr@ctlist}}

\newcommand{\smartcopyright}[1]{\begingroup%
	\pagebreak%
	\thispagestyle{empty}%
	\footnotesize%
	\par\vspace*{\fill}%
	\hspace{-1.1cm}%
	\parbox{10cm}{%
		\noindent%
		Copyright \textcopyright\ \the\year\ \applytolist[, ]{}{\@author}\\[0.5cm]%
		{#1}%
	}%
\endgroup}

\newcommand{\smartinitialepigraph}[2]{\begingroup
	\cleardoublepage
	\thispagestyle{empty}
	\vspace*{2cm}
	\epigraph{#1}{#2}
	\cleardoublepage
\endgroup}

\let\oldepigraph\epigraph
\renewcommand{\epigraph}[2]{\oldepigraph{#1}{\spacedlowsmallcaps{#2}}}

\makeatother

%
% MARGINNOTE
%

%
% There are three commands for placing elements in the margin:
%
% \marginnote{<TEXT>}
% Just like \marginpar but with correct formating
%
% \marginfig{<SHORT CAPTION>}{<FIGURE CONTENT>}{<LONG CAPTION>}{\label{<LABEL>}}
% \margintbl{<SHORT CAPTION>}{<TABLE CONTENT>}{<LONG CAPTION>}{\label{<LABEL>}}
% Places a figure/table in the margin
% - <SHORT CAPTION> is the caption as it occurs in the list of figures/tables
% - <FIGURE CONTENT>/<TABLE CONTENT> is the actual figure/table content
% - <LONG CAPTION> is the caption below the figure
% - <LABEL> is the label -- we decided to let you explicitly type "\label" here
% as this allows LaTeX editors to index the label and provide auto completion.
%

% Define the "marginnote" command. For compatibility reasons
% we do not override marginpar.
\strictpagecheck
\newcommand{\marginnote}[1]{\hspace*{0pt}\marginpar{
	\microtypesetup{disable}%
	\slshape\footnotesize%
	\vspace*{-0.725em}%
	\checkoddpage%
	\ifoddpage%
		\RaggedRight%
	\else%
		\RaggedLeft%
	\fi%
	{#1}%
	\microtypesetup{enable}%
}\ignorespaces}

\newcommand{\marginfloat}[8]{\marginnote{{#1}%
\par\refstepcounter{#4}\textcolor{margincaption}{#5 #6}: {#2}#3}%
\addcontentsline{#7}{#4}{\protect\numberline {#6}{\ignorespaces #8}}\ignorespaces}

\newcommand{\marginfig}[4]{\marginfloat{#2}{#3}{#4}{figure}{Figure}{\thefigure}{lof}{#1}}
\newcommand{\margintbl}[4]{\marginfloat{#2}{#3}{#4}{table}{Table}{\thetable}{lot}{#1}}

%
% PAGE STYLE (headers, footers)
%

\makeatletter

% Deactivate capitalization of headers.
\nouppercaseheads

% Define the new "berlin" style.
\makepagestyle{berlin}

% Display the page number in the footer.
\makeevenfoot{berlin}{\thepage}{}{}
\makeoddfoot{berlin}{}{}{\thepage}

% Override the chapter style with the unfinished "berlin" style
% to eliminate the header and to make sure a consistent footer
% style is used.
\copypagestyle{chapter}{berlin}

% Define the "berlin" style
\makepsmarks{berlin}{%
	\def\chaptermark##1{%
	\markboth{\memUChead{%
		\ifnum \c@secnumdepth >\m@ne
		\if@mainmatter
			\@chapapp\ \thechapter. \ %
		\fi
		\fi
		##1}}{}}%
	\def\tocmark{\markboth{\memUChead{\contentsname}}{\memUChead{\contentsname}}}%
	\def\lofmark{\markboth{\memUChead{\listfigurename}}{\memUChead{\listfigurename}}}%
	\def\lotmark{\markboth{\memUChead{\listtablename}}{\memUChead{\listtablename}}}%
	\def\bibmark{\markboth{\memUChead{\bibname}}{\memUChead{\bibname}}}%
	\def\indexmark{\markboth{\memUChead{\indexname}}{\memUChead{\indexname}}}%
	\def\sectionmark##1{%
	\markright{\memUChead{%
		\ifnum \c@secnumdepth > \z@
		\thesection. \ %
		\fi %
		##1}}}}
\makepsmarks{berlin}{%
	\createmark{chapter}{left}{nonumber}{}{}
	\createmark{section}{right}{shownumber}{}{\ }
	\createplainmark{toc}{both}{\contentsname}
	\createplainmark{lof}{both}{\listfigurename}
	\createplainmark{lot}{both}{\listtablename}
	\createplainmark{bib}{both}{\bibname}
	\createplainmark{index}{both}{\indexname}
	\createplainmark{glossary}{both}{\glossaryname}
}

% Discriminate between "oneside" and "twoside" documents.
\if@twoside
	\makeevenhead{berlin}{\spacedlowsmallcaps{\leftmark}}{}{}
	\makeoddhead{berlin}{}{}{\spacedlowsmallcaps{\rightmark}}
\else
	\makeoddhead{berlin}{\spacedlowsmallcaps{\rightmark}}{}{}
\fi

% Use the berlin style.
\pagestyle{berlin}

\makeatother

%
% FOOTNOTES
%

\footmarkstyle{\textsc{#1}}
\setlength{\footmarkwidth}{-0.5em}
\setlength{\footmarksep}{0.5em}
\setlength{\footparindent}{0em}

%
% EPIGRAPHS
%
\makeatletter

% Make the epigraph span two third of the text width
\setlength{\epigraphwidth}{0.66\textwidth}

% Disable the rule below the epigraph
\setlength{\epigraphrule}{0pt}

% Use the normal font size
\epigraphfontsize{\normalsize}

% Style the epigraph itself -- use italic text and
% flush it to the right
\newenvironment{epigraphstyle}{
%	\microtypesetup{disable}%
	\small\slshape%
%	\RaggedLeft%
}{%
%	\microtypesetup{enable}%
}
\epigraphtextposition{epigraphstyle}

% Style the epigraph source -- prepend it with a em-dash
% and make sure no indentation is added after the epigraph
\newenvironment{epigraphsourcestyle}%
{%#
	--- \small\raggedleft\color{smartred}%
}{%
	\@afterindentfalse\@afterheading%
}
\epigraphsourceposition{epigraphsourcestyle}

\makeatother

%
% TITLEPAGE
%

\makeatletter
\newcommand\thesistype[1]{\renewcommand\@thesistype{#1}}
\newcommand\@thesistype{}

\newcommand\discipline[1]{\renewcommand\@discipline{#1}}
\newcommand\@discipline{}

\newcommand\institution[1]{\renewcommand\@institution{#1}}
\newcommand\@institution{}

\newcommand\supervisors[1]{\renewcommand\@supervisors{#1}}
\newcommand\@supervisors{}

\newcommand\internalid[1]{\renewcommand\@internalid{#1}}
\newcommand\@internalid{}

\newcommand{\applytolist}[3][{,\\\vspace{0.1cm}}]{%
	\def\nextitem{\def\nextitem{#1}}%
	\@for \el:=#3\do{\nextitem{#2{\el}}}%
}

\newcommand{\smarttitle}{
{\begingroup
\thispagestyle{empty}
\makeatletter
\raggedright
\vfill
{\LARGE \spacedallcaps{\@thesistype} \\}
{\Large \spacedlowsmallcaps{\@discipline}\\}
\vfill
{\Huge \textcolor{smartred}{\spacedallcaps{\@title}}\\}
\vfill
{\LARGE \applytolist{\spacedallcaps}{\@author}\\}
\vfill
{\Large \applytolist{\itshape}{\@institution}}
\vfill
{\Large \spacedlowsmallcaps{Supervised by}\\}
\vspace{0.1cm}
{\large \applytolist{\spacedallcaps}{\@supervisors}}
\vfill
\vfill
{\Large \spacedlowsmallcaps{\@date} \hfill \spacedlowsmallcaps{\@internalid}}
\vfill
\makeatother
\endgroup}}

\makeatletter
%
% PAGE LAYOUT
%
\settypeblocksize{23.45cm}{11.85cm}{*}
%\setlength{\textheight}{23.45cm}
%\setlength{\textwidth}{11.85cm}
\setlrmargins{3.141592653589793cm}{*}{*}
\setulmargins{*}{*}{*}
\setmarginnotes{0.8cm}{3.5cm}{1cm}
\checkandfixthelayout


% Contains some packages which are commonly used in a thesis.
% Note that these are optional.
\usepackage{csquotes} % Context sensitive quotation facilities. Recommended by babel and should be loaded before babel.

\usepackage[english]{babel} % Sets the language used. Essential for proper hyphenation. Translates key words like 'figure' or 'table'. Note that 'english' is American English.

\usepackage{latexsym,amsmath,amssymb,amsthm,amscd} % Provides math enviroments, math symbols and more things useful for math.

\usepackage{mathtools} % Enhances amsmath and provides further mathematical tools.

\usepackage{graphicx} % Provides inclusion of graphics and a proper interface for '\in­clude­graph­ics'.
% Su­per­sedes 'epsfig' and 'graphics'.

\usepackage{enumitem} % Provides control over list environments. Supersedes 'enumerate', which gives enumerate environment an optional argument which determines the style in which the counter is printed.

%\let\newfloat\undefined % Hack to make floatrow work with memoir.
%\usepackage{floatrow} % Handles alignment of floats (figures), with centering as default.

\usepackage{xspace} % Ugly hack that sometimes helps with otherwise missing spaces.

\usepackage[backgroundcolor=white,linecolor=smartblue,bordercolor=smartblue,textsize=footnotesize]{todonotes} % Todo notes.

% Use sans-serif font in todos to clearly distinguish it from other text
\makeatletter
\renewcommand{\todo}[2][]{\@bsphack\@todo[#1]{\sffamily{#2}}\@esphack\ignorespaces}
\makeatother

\usepackage[style=alphabetic,backend=biber]{biblatex} % Bibliography.

%\usepackage{algorithm} % Algorithms.
% Use default Memoir floats for algorithms
\newcommand{\algorithmname}{Algorithm}
\newcommand{\listalgorithmname}{List of Algorithms}
\newlistof{listofalgorithms}{loa}{\listalgorithmname}
\newfloat[chapter]{algorithm}{loa}{\algorithmname}
\newfixedcaption{\falgcaption}{algorithm}
\newlistentry[chapter]{algorithm}{loa}{0}
\cftsetindents{algorithm}{0em}{2.3em}
\makeatletter
\g@addto@macro\insertchapterspace{\addtocontents{loa}{\protect\addvspace{10pt}}}
\makeatother

\usepackage{algpseudocode} % Algorithms.

\usepackage{varioref} % Automatically locates references on other pages. Load before hyperref.

%\usepackage[hidelinks]{hyperref} % Clickable references.
\usepackage[pdfencoding=auto,pdfborderstyle={/S/U/W 1},linkbordercolor=ref,colorlinks,linkcolor=ref,citecolor=cite,urlcolor=url]{hyperref} % Clickable references.

\usepackage[all]{hypcap} % Anchors links to the beginning of their respective floats. Load after hyperref.

\usepackage[noabbrev,capitalize,nameinlink]{cleveref} % Names references automatically. Load after hyperref.

\usepackage[toc,acronym]{glossaries} % Glossary.

\usepackage{booktabs}
\usepackage{multirow}

\usepackage{pgfplots}
\pgfplotsset{compat=1.8}
\pgfplotscreateplotcyclelist{smart}{%
  tangoplum,semithick,every mark/.append style={fill=tangoplum!80!black},mark=*\\%
  tangogreen,semithick,every mark/.append style={fill=tangogreen!80!black},mark=square*\\%
  tangoorange,semithick,every mark/.append style={fill=tangoorange!80!black},mark=otimes*\\%
  tangoblue,semithick,mark=star\\%
  tangobutter,semithick,every mark/.append style={fill=tangobutter!80!black},mark=diamond*\\%
  tangored,semithick,every mark/.append style={solid,fill=tangored!80!black},mark=*\\%
}
\pgfplotsset{cycle list name=smart}



% Contains some macros which are commonly used in a thesis.
% Note that these are optional.
% Abbreviations
\newcommand*{\eg}{e.\,g.\@\xspace}
\newcommand*{\ie}{i.\,e.\@\xspace}
\newcommand*{\cf}{cf.\@\xspace}
\newcommand*{\etal}{et.\@ al.\@\xspace}
%\newcommand*{\etc}{etc.\@\xspace}
\makeatletter
\newcommand*{\etc}{%
  \@ifnextchar{.}%
    {etc}%
      {etc.\@\xspace}%
}
\makeatother

% Sets
\newcommand*{\R}{\mathbb{R}}
\newcommand*{\Q}{\mathbb{Q}}
\newcommand*{\Z}{\mathbb{Z}}
\newcommand*{\N}{\mathbb{N}}

% Matrix/vector operations
\newcommand*{\transpose}[1]{{#1^{\top}}}
\newcommand*{\inverse}[1]{{#1^{-1}}}
\newcommand*{\conjugate}[1]{{#1^{\ast}}}
\newcommand*{\pseudoinverse}[1]{{#1^{+}}}

% Stochastics
\newcommand*{\probability}[1]{{\mathrm{Pr}{#1}}}
\newcommand*{\expectation}[1]{{#1}}

% Optimization
\DeclareMathOperator*{\argmax}{arg\,max}
\DeclareMathOperator*{\argmin}{arg\,min}
\DeclareMathOperator{\lb}{lb}
\DeclareMathOperator{\sign}{sgn}

% These packages are only used in the demo and not necessarily
% required for smart-thesis.
\usepackage{blindtext}

% Loads bibliography for citations.
\addbibresource{demo-bibliography.bib}

% Starts creation of a glossary.
\makeglossaries
\newglossaryentry{mjollnir}{
  name=Mjölnir,
  description={the hammer of Thor}}

\newglossaryentry{wolf}{
  name=wolf,
  plural=wolves,
  description={an awesome animal and a subspecies of \emph{Canis lupus}}}
  


% Define all the metadata to be listed in \smarttitle and \smartcopyright
\thesistype{Beachelor Thesis}
\discipline{Kognitive Informatik}
\title{Systematic Generation of Unit-Tests with different LLMs and prompt-design}
\author{Richard Pamies}
\institution{University of Bielefeld}
\supervisors{Prof.\@~Dr.-Ing.\@~Benjamin Paaßen,M.Sc.\@~Jesper Dannat}

\begin{document}

\frontmatter

\smarttitle

\newpage
\tableofcontents

\mainmatter

\chapter{Introduction}
Unit-test are really important and useful. But most often those tests don't get written because of laziness, so the automatic generation of those tests arise as a possible solution.
\\
Large Language Models (LLMs) offer for the the possibility of creating useful unit-tests. But those generated test are flawed because those LLMs dont work with reason, but instead the statistically most probable answer.
\\
Prior work  has tried under different circumstances to solve this problem.[paper 1,2,3].
\\
My goal in this work is to:\\
\begin{enumerate}
%\itemsep-1.5em
\item measure the code coverage and statement coverage of generated unit-test
\item generate these unit-tests with different LLMs
\item generate these unit-tests with different prompt design strategies
\end{enumerate}
and then at the end i want to compare the different approaches.

\chapter{Background and related work}
\section{Unit-tests}
Unit-tests are tests written for already existing code to make sure the implemented functionalities still work after further progress on the code is done. The usefulness of unit-tests is not controversial. Rather the time it takes to write them and the metrics used to test their usefulness. 
Code coverage and statement coverage are two well-known and often used metrics in the context of unit-tests. There are more metrics like the mutation-score but code and statement coverage are the most implemented and available.
For all metrics concerning unit-test and above it is important to keep in mind that there a no metrics that can be used to guarantee bug-free code. It is more an indication of where more tests might be needed rather than making any statement about the quality and conceptual coverage that the tests provide for the actual code.
So in the end there are no perfect, singular metrics that can be used to assess the quality and conceptual coverage of unit-tests. Experience and the combined use of several different metrics to arrive at an indication is the best way to create high-quality unit-tests.
\subsection{code coverage}
Code coverage is a measure of how much of a program's source code is executed during testing. It works by tracking which lines of code, functions, or branches are run when tests are performed.  
\subsection{statement coverage}
Statement coverage on the other hand measures the percentage of executable statements in a program that are executed during testing. However, it has limitations as it doesn't account for different paths through decision points
\subsection{mutation coverage}
Mutation score is a measure of test effectiveness that assesses how well tests detect intentional code changes made by mutations. It works by automatically introducing small alterations to the source code, such as changing operators or variable values, and then running the test. The score is the percentage of mutations detected by failing tests. Mutation score is important because it evaluates the quality and sensitivity of tests, revealing weaknesses in test cases that other metrics might miss. A high mutation score suggests that tests are good at catching potential bugs. However, generating and testing mutations can be computationally expensive.
\subsection{cyclomatic code complexity}
Cyclomatic complexity is a software metric that measures the number of linearly independent paths through a program's source code. It works by analyzing the control flow graph of a program, counting decision points like if statements and loops. A higher cyclomatic complexity indicates more complex code. This metric is important because it helps identify overly complicated functions or methods that may be difficult to test, maintain, or understand. 
\section{LLMs}
Large Language Models(LLMs) are computational models that normaly used an enormous data set for learning and are able to achieve general-purpose language generation and other tasks like categorization and other multi-modal tasks. An LLM is only as good as its architecture and its training data. 
In the case of unit-tests
Generating unit-tests for mainstream code might be sufficient. But it is really easy to write code which covers niche subjects. And in those cases the LLM would need critical thinking, understanding and reasoning to fully understand the code and write useful and encompassing unit-tests.
\subsection{GPT}
ChatGPT is an advanced conversational AI model developed by OpenAI. It works by using a large language model trained on diverse internet text to generate human-like responses to user inputs. ChatGPT can engage in dialogue, answer questions, and assist with various tasks across multiple domains.
\subsection{Open source models}
\section{related work}

\chapter{Methods}
This 
high level description
what it should accomplish
detailed description for reproduction for experienced user
use illustrating diagrams as much as possible
(no source code, but perhaps pseudo code)
provide a good reason for each design choice(source or math)
use a running example to illustrate the steps
\section{LLMs}
I used chat-gpt(GPT) 3.5, GPT 4o, OPEN_SOURCE and OPEN_SOURCE. 
I used GPT because of its accessibility and prominence and to expand my range i also included the open source models
For all of these i used an API endpoint to send my prompts to and in the case of the open source models i used my own desktop PC for running the models and hosting the API endpoint.
These LLMs are the core component that return generated code through prompts and are each more or less capable in the different capabilities necessary to generate unit-tests. 
\section{Prompt-design}
A strong leverage to control the resulting output quality of LLMs is prompt-design. Prompt-design describes the mostly iterative, model-specific and experience driven design process of how the textual input prompt is worded and structured to achieve the optimal result.
In my experiments i validated my prompts with an example exercise and GPT 3.5. 
My prompt, that was send in the end to the model, consisted of four parts that were togglable:
\begin{enumerate}
%\itemsep-1.5em
\item My static initialization prompt, which instructed the model to generate a unit-test with the information's available.
\item A Task description, that was copied from the Task description and, if necessary, slightly adjusted.
\item The code of the example solution, 
\item The function name and signature,
\item Unit-test examples of the example unit-test,
\item The error message in the case of failure, so that errors could in the best case be specifically addressed and fixed.
\end{enumerate}


There were two main static prompts needed for my application:
\begin{itemize}
%\itemsep-1.5em
\item The instruction prompt, that combined with the dynamically read task attributes forms the prompt for the model. 

\\ \noindent\fbox{\begin{minipage}{\textwidth}
\paragraph{Instruction prompt}
The things above are the parts of the description of a programming task.
Now write me a Unit test in python for to validate the code that was given above.
Think of the imports. 
Be thorough.
\end{minipage}}\\
I was forced to re-design the prompt a few times because GPT 3.5 was  suddenly forgetting, or misunderstanding important aspects. Examples were that suddenly no code was generated, only a textual directive on how to write a unit-test.

\item The role description, which describes the role of the model in the interaction. I mainly used the role description to control the output of the model.
\\ \noindent\fbox{\begin{minipage}{\textwidth}
\paragraph{Role description:}
You are an outstanding programmer, 
but secluded and efficient. As a professional programmer you answer as concise and precise as possible without any unnecessary words
\end{minipage}}\\
This resulted in comparatively reliable results where the amount of unnecessary  text was reduced.
 
\end{itemize}

\section{Training-data}
I received 22 exercises which were exercises for the python entry course of Prof. Paaßen. Each of the 22 folders contained at least a Task description,an example unit-test and an example solution for the code. 
\section{Pre-processing}
I prepared each of the 22 Tasks manually by creating a "prompt" text-file and pasting the information's that are necessary in defined categories. That enabled me to then parse the information's for each task programmatically.
\section{Post-processing and validation}
When receiving the code i wrote a small parser that extracted the code from the resulting string and put that into a temporary file to run the unit-test.
The unit-test was then tested and the return value determined if a new attempt was to be made.
\section{Strategies for failures}
If the generated unit-test did not execute correctly although it should have, a new prompt is send to the model accompanied by the produced error message. 

If after that the result still doesn't compile, another prompt is send with the query to remove the faulty line. That is done to at least have a compiling test result in the end.
\\The structured prompt could look like this:


\section{provide overview diagram illustrating my steps}

\chapter{Implementation}
\section{ Code examples}

\chapter{Experiments}
\section{baselines}
\section{what i want to show}
\section{how i want to show it}
\section{baselines and assumptions}
\section{what was my initial assumption}
\section{Table for experiments}
\section{Discussion of the results}

\chapter{Conclusion}
\section{re-telling of the paper}
\section{high-level overview of experimental results}
\section{Limitations}
\section{what future work could accomplish}
\section{final words}
		
\newpage
			
\chapter{latex features}
\\ \noindent\fbox{\begin{minipage}{\textwidth}
\paragraph{Lemma} Let the coefficients of the polynomial
$$a_0+a_1x+a_2x^2+\cdots+a_{m-1}x^{m-1}+x^m$$ be integers. 
Then any real root of the polynomial is either integral or irrational.
\end{minipage}}\\\\\\

sparse\footnote{This is a footnote}

Did you know? The discrete fourier transformation shown in equation \ref{eqn:dft} is the backbone of the modern information-society.
		
\begin{align}
	f_m  \label{eqn:dft}
\end{alig
n}
For words of science, see \cite{botsch2010polygon}. Unfortunately, that book has nothing about \glspl{wolf}.

\backmatter
\printglossaries
\printbibliography[heading=bibintoc]

\end{document}

